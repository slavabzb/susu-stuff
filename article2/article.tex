\documentclass[runningheads,a4paper]{llncs}

\usepackage{amssymb}
\setcounter{tocdepth}{3}
\usepackage{graphicx}

\usepackage{url}
\urldef{\mailsa}\path|{anatoly.panyukov, avaksa, vyacheslav.bezborodov}@gmail.com|
\urldef{\mailsb}\path||
\urldef{\mailsc}\path||
\newcommand{\keywords}[1]{\par\addvspace\baselineskip
\noindent\keywordname\enspace\ignorespaces#1}

\begin{document}

\mainmatter  % start of an individual contribution

% first the title is needed
\title{Parallel implementation of the ellipsoid method for optimization problems of large dimension\thanks{The work is supported by RFBR grant 16-31-00108\textbackslash 16 "Solutions to fundamental issues of usability of extended and arbitrary-precision arithmetics".}}

% a short form should be given in case it is too long for the running head
\titlerunning{Parallel implementation of the ellipsoid method}

% the name(s) of the author(s) follow(s) next
%
% NB: Chinese authors should write their first names(s) in front of
% their surnames. This ensures that the names appear correctly in
% the running heads and the author index.
%
\author{Anatoly Panyukov\and Valentin Golodov\and Vyacheslav Bezborodov}
%
\authorrunning{Parallel implementation of the ellipsoid method}
% (feature abused for this document to repeat the title also on left hand pages)

% the affiliations are given next; don't give your e-mail address
% unless you accept that it will be published
\institute{South Ural State University, 76, Lenin prospect, Chelyabinsk, 454080, Russia\\
\mailsa
\mailsb
\mailsc
\url{}}

%
% NB: a more complex sample for affiliations and the mapping to the
% corresponding authors can be found in the file "llncs.dem"
% (search for the string "\mainmatter" where a contribution starts).
% "llncs.dem" accompanies the document class "llncs.cls".
%

\toctitle{Parallel implementation of the ellipsoid method}
\tocauthor{A.Panyukov, V.Golodov, V.Bezborodov}
\maketitle


\begin{abstract}
A parallel implementation of the ellipsoid method is discussed in the paper. The method is important in theoretical researches but for practical issues it has some lacks such as low rate of convergence and round-error sensitiveness.

Authors suggest reducing the influence of the low rate of convergence at the expense of parallel computations. To prevent the loss of significance authors use arbitrary precision floating point numbers.

Parallel matrix operations were implemented as C shared library using OpenMP interface. The MPFR library was used for multiple-precision floating-point computations with correct rounding.

\keywords{linear optimization, ellipsoids method, parallel programming, arbitrary precision}
\end{abstract}


\section{Introduction}

Consider a method that uses approximation of the localization sets. The method is based on the following geometric observation.

Let $H$ be a positive-definite symmetric $(n\times n)$-matrix. Consider the ellipsoid $$E(H,\overline{x})=\left\{x\in\mathbb{R}^n|\langle H^{-1}(x-\overline{x}),x-\overline{x}\rangle\leq 1\right\}.$$ Choose the direction $g\in\mathbb{R}^n$ and consider the half of the ellipsoid defined by corresponding hyperplane: $$E_+=\left\{x\in E(H,\overline{x})|\langle g,\overline{x}-x\rangle\geq 0\right\}.$$ Obviously this set is contained in another ellipsoid having a volume less than a volume of ellipsoid $E(H,\overline{x})$.

\begin{lemma}
\proof\cite{nesterov}
Lets define

$$
\overline{x}_+=\overline{x}-\frac{1}{n+1}\cdot\frac{H_kg_k}{\langle H_kg_k,g_k\rangle^{1/2}},
$$

$$
H_+=\frac{n^2}{n^2-1}\left(H-\frac{2}{n+1}\cdot\frac{H_kg_kg_k^TH_k}{\langle H_kg_k,g_k\rangle}\right).
$$

Then $E_+\subset E(H_+,\overline{x}_+)$ and

$$
vol_nE(H_+,\overline{x}_+)\leq\left(1-\frac{1}{(n+1)^2}\right)^{n/2}vol_nE(H,\overline{x}).
$$

\end{lemma}

The volume of the ellipsoid $E(H_+,\overline{x}_+)$:

$$
\frac{vol_nE(H_+,\overline{x}_+)}{vol_nE(H,\overline{x})}=\left[\frac{\det H_+}{\det H}\right]^{1/2}=\left[\left(\frac{n^2}{n^2-1}\right)^n\frac{n-1}{n+1}\right]^{1/2}=
$$

$$
=\left[\frac{n^2}{n^2-1}\left(1-\frac{2}{n+1}\right)^\frac{1}{n}\right]^{n/2}\leq\left[\frac{n^2}{n^2-1}\left(1-\frac{2}{n(n+1)}\right)\right]^{n/2}=
$$

$$
=\left[\frac{n^2(n^2+n-2)}{n(n-1)(n+1)^2}\right]^{n/2}=\left[1-\frac{1}{(n+1)^2}\right]^{n/2}.
$$

So $E(H_+,\overline{x}_+)$ is an ellipsoid of minimal volume which contains a half of the original ellipsoid $E_+$.

These statements can be represented as \textit{an algorithm of the ellipsoid method}.

\fbox{\parbox{\textwidth}{
\textbf{Ellipsoid method}
\begin{enumerate}
\item Choose $y_0\in\mathbb{R}$ and $\mathbb{R}>0$ so that euclidian ball $B_2(y_0,\mathbb{R})\supseteq Q$. Let $H_0=\mathbb{R}^2\cdot I_n$, $y^*=0$, $f^*=\inf$.
\item $k$-th iteration ($k\geq 0$):
$$
g_k=\left\{\begin{array}{ll}
g(y_k), & \mbox{if } y_k\in Q,\\
\hat{g}(y_k), & \mbox{if } y_k \notin Q,
\end{array}
\right.
$$
$$
y_{k+1}=y_k-\frac{1}{n+1}\cdot\frac{H_kg_k}{\langle H_kg_k,g_k\rangle^{1/2}},
$$
$$
y^*=y_{k+1},f^*=f(y^*),\mbox{ if }f(y_{k+1})<f^*,
$$
$$
H_{k+1}=\frac{n^2}{n^2-1}\left(H_k-\frac{2}{n+1}\cdot\frac{H_kg_kg_k^TH_k}{\langle H_kg_k,g_k\rangle}\right).
$$
\end{enumerate}
}
}

Lets evaluate the effectiveness of the ellipsoid method. Let $Y=\{y_k\}^\infty_{k=0}$ and let $X$ be a feasible part of $Y$ subsequence: $$X=Y\cap Q.$$ Let $f^*_k=\min_{0\leq j\leq k}f(x_j)$.

\begin{theorem}
\proof\cite{nesterov}
Let $f$ be a Lipschitz function with domain $B_2(x^*,\mathbb{R})$ bounded by a Lipschitz constant $M$. Then for $i(k)>0$ the following inequality is true: $$f^*_{i(k)}-f^*\leq M\mathbb{R}\left(1-\frac{1}{(n+1)^2}\right)^{k/2}\cdot\left[vol_nB_0(x_0,\mathbb{R})vol_nQ\right]^{1/n}.$$
\end{theorem}


\section{Implementation}

C is a general-purpose, imperative computer programming language, supporting structured programming, lexical variable scope and recursion. By design, C provides constructs that map efficiently to typical machine instructions, and therefore it has found lasting use in applications that had formerly been coded in assembly language, including operating systems, as well as various application software for computers ranging from supercomputers to embedded systems~\cite{kern}.

Since C doesn't support complex data structures and provide low-level mechanism of grouping variables called plain old data structure (POD) for this, we need to define a data type which represents matrices in source code. It can be achieved by using C \textit{struct} keyword (see Listing 1).

\begin{verbatim}
/**
 * Matrix structure.
 */
struct mtx
{
    mpfr_t* storage;     ///!< internal storage
    size_t nrows;        ///!< number of rows
    size_t ncols;        ///!< number of columns
};
\end{verbatim}
%
\noindent
{\small\textbf{Listing 1.} Matrix representation: \textit{storage} is a pointer to allocated memory block, \textit{nrows} and \textit{ncols} are used for pointer offset.\bigskip}

There are at least two methods to declare N by M 2D array in C. The first one involves N allocation requests, one for each row, plus one request for array of row arrays. The bucket locations in individual rows are contiguous, but rows are not necessarily contiguous in heap space. By using this method element accessing can be done with \texttt{array[i][j]} syntax which is more preferable for humans. The second one is considered to be memory efficient: there is the only one N by M array allocation. Really this is a large 1-dim array of values onto which we will map 2D accesses. In that case we cannot use \texttt{array[i][j]} syntax because the compiler has no idea where the next row starts within this chunk of heap space, so must use single index value that is calculated using row and column index values and the column dimension.

Note that the type of elements in internal storage is \texttt{mpfr\_t}. It is a special data type from The GNU MPFR Library\footnote{http://www.mpfr.org/} -- a portable library written in C for arbitrary precision arithmetic on floating-point numbers. It is based on the GNU MP library. It aims to provide a class of floating-point numbers with precise semantics. The main characteristics of MPFR, which make it differ from most arbitrary precision floating-point software tools, are:
\begin{itemize}
\item the MPFR code is portable, i.e., the result of any operation does not depend on the machine word size \texttt{mp\_bits\_per\_limb} (64 on most current processors);
\item the precision in bits can be set exactly to any valid value for each variable (including very small precision);
\item MPFR provides the four rounding modes from the IEEE 754-1985 standard, plus away-from-zero, as well as for basic operations as for other mathematical functions. 
\end{itemize}

In particular, with a precision of 53 bits, MPFR is able to exactly reproduce all computations with double-precision machine floating-point numbers (e.g., double type in C, with a C implementation that rigorously follows Annex F of the ISO C99 standard and \texttt{FP\_CONTRACT} pragma set to OFF) on the four arithmetic operations and the square root, except the default exponent range is much wider and subnormal numbers are not implemented (but can be emulated).

In order to manipulate represented matrix POD structure, we provide simple basic interface to our library. Some methods are shown in Listing 2.

\begin{verbatim}
// memory handling
int mtx_init(struct mtx* const m,
        size_t rows, size_t columns, mpfr_prec_t prec);

int mtx_clear(struct mtx const m);

// matrix I/O
int mtx_fprint(FILE* stream, struct mtx const m);
int mtx_fscan(FILE* stream, struct mtx m, char const* delim);

// assignment
int mtx_fill(struct mtx m, mpfr_t val, mpfr_t diagval);
int mtx_fill_d(struct mtx m, double val, double diagval);

// copying
int mtx_copy(struct mtx rop, struct mtx const op);

// multiplication
int mtx_mul(struct mtx rop,
        struct mtx const op1, struct mtx const op2);

int mtx_mulval(struct mtx rop, struct mtx const op1, mpfr_t op2);

// addition
int mtx_add(struct mtx rop,
        struct mtx const op1, struct mtx const op2);

// transposition
int mtx_tr(struct mtx rop, struct mtx const op);
\end{verbatim}
%
\noindent
{\small\textbf{Listing 2.} Matrix library interface.\bigskip}

Let us give a short brief. All the functions are grouped according to a certain kind of operations:
\begin{itemize}
\item\textbf{memory handling} -- matrix creation/destroying with memory allocation/freeing (note the precision parameter; it is the number of bits used to represent the significand of a floating-point number; the precision can be any integer between \texttt{MPFR\_PREC\_MIN} and \texttt{MPFR\_PREC\_MAX});
\item\textbf{input/output facilities} -- reading/writing matrices from/to different sources;
\item\textbf{assignment operations} -- filling matrices by certain value (note \texttt{diagval} parameter which is used for diagonal elements filling if matrix is square);
\item\textbf{basic operations} -- multiplication (matrix product), multiplication by value, addition, transposition and copying.
\end{itemize}

Note the return value of each function. The library is designed to report about a problem by returning 0 on success and non-zero otherwise.

Due to specific nature, the most matrix operations can be performed in parallel without any data races, resource acquiring and minimal data sharing. That's why the major part of the operations from Listing 2 are parallelized to be efficient.

There are a lot of approaches to organize parallelism. One of the most popular, modern and flexible is OpenMP (Open Multi-Processing). OpenMP is an application programming interface (API) that supports multi-platform shared memory multiprocessing programming in C, C++, and Fortran,\cite{gagne} on most platforms, processor architectures and operating systems, including Solaris, AIX, HP-UX, Linux, OS X, and Windows. It consists of a set of compiler directives, library routines, and environment variables that influence run-time behavior.

OpenMP uses a portable, scalable model that gives programmers a simple and flexible interface for developing parallel applications for platforms ranging from the standard desktop computer to the supercomputer.

An application built with the hybrid model of parallel programming can run on a computer cluster using both OpenMP and Message Passing Interface (MPI), such that OpenMP is used for parallelism within a (multi-core) node while MPI is used for parallelism between nodes. There have also been efforts to run OpenMP on software distributed shared memory systems,\cite{costa} to translate OpenMP into MPI\cite{wang,basumallik} and to extend OpenMP for non-shared memory systems.

Let us to show a way how our matrix library encapsulates creation a number of workers by using OpenMP API behind the scene. Lets look through the matrix multiplication method. The interface of the method is shown below.

\begin{verbatim}
int mtx_mul(struct mtx rop,
        struct mtx const op1, struct mtx const op2)
\end{verbatim}

The function accepts two operands \texttt{op1} and \texttt{op2} and saves the result of the operation in \texttt{rop}. Firstly, we need to ensure that dimensions of the accepted values is correct.

\begin{verbatim}
if (rop.nrows != op1.nrows || rop.ncols != op2.ncols)
    return -1;
	
if (op1.ncols != op2.nrows)
    return -1;
\end{verbatim}

Next, we marks internal loops as \textit{OpenMP parallel region}. Work-sharing constructs used to specify how to assign independent work to one or all of the threads. Directives \texttt{omp for} or \texttt{omp do} are used to split up loop iterations among the threads, also called loop constructs.

\begin{verbatim}
size_t i, j, k;

#pragma omp parallel for
    shared(rop) private(i,j,k) schedule(static)
    
    for (i = 0; i < op1.nrows; ++i)
    {
        for (j = 0; j < op2.ncols; ++j)
        {
            mpfr_t* const prop = rop.storage + i * rop.ncols + j;	
            mpfr_set_ui(*prop, 0, MPFR_RNDN);
            
            for (k = 0; k < op1.ncols; ++k)
            {
                // do multiplication ...
            }
        }
    }
\end{verbatim}

Since OpenMP is a shared memory programming model, most variables in OpenMP code are visible to all threads by default. But sometimes private variables are necessary to avoid race conditions and there is a need to pass values between the sequential part and the parallel region (the code block executed in parallel), so data environment management is introduced as data sharing attribute clauses by appending them to the OpenMP directive. Some types of clauses are:

\begin{itemize}
\item\textbf{shared} -- the data within a parallel region is shared, which means visible and accessible by all threads simultaneously. By default, all variables in the work sharing region are shared except the loop iteration counter.

\item\textbf{private} -- the data within a parallel region is private to each thread, which means each thread will have a local copy and use it as a temporary variable. A private variable is not initialized and the value is not maintained for use outside the parallel region. By default, the loop iteration counters in the OpenMP loop constructs are private.
\end{itemize}

One of the \textit{scheduling clauses} is \texttt{schedule(type, chunk)}. This is useful if the work sharing construct is a do-loop or for-loop. The iteration(s) in the work sharing construct are assigned to threads according to the scheduling method defined by this clause. By using the \texttt{static} schedule, all the threads are allocated iterations before they execute the loop iterations. The iterations are divided among threads equally by default. However, specifying an integer for the parameter chunk will allocate chunk number of contiguous iterations to a particular thread.

Finally, the innermost loop do the multiplication routine.

\begin{verbatim}
mpfr_t* const pop1 = op1.storage + i * op1.ncols + k;
mpfr_t* const pop2 = op2.storage + k * op2.ncols + j;

mpfr_t tmp;
mpfr_init2(tmp, prec);

mpfr_mul(tmp, *pop1, *pop2, MPFR_RNDN);
mpfr_add(*prop, *prop, tmp, MPFR_RNDN);

mpfr_clear(tmp);
\end{verbatim}

There are a lot of specific functions from MPFR library here, such as multiplication and addition. Note using the \texttt{MPFR\_RNDN} parameter. It is called \textit{rounding mode}. The following five rounding modes are supported by GNU MPFR Library:

\begin{enumerate}
\item\textbf{MPFR\_RNDN} -- round to nearest (roundTiesToEven in IEEE 754-2008),
\item\textbf{MPFR\_RNDZ} -- round toward zero (roundTowardZero in IEEE 754-2008),
\item\textbf{MPFR\_RNDU} -- round toward plus infinity (roundTowardPositive in IEEE 754-2008),
\item\textbf{MPFR\_RNDD} -- round toward minus infinity (roundTowardNegative in IEEE 754-2008),
\item\textbf{MPFR\_RNDA} -- round away from zero. 
\end{enumerate}

That is the way how the most matrix library functions handle matrices of large dimension. Of course, not all the functions can be performed in parallel (matrix I/O facilities is still consequent for obvious reasons).

\subsection{Checking the PDF File}

Kindly assure that the Contact Volume Editor is given the name and email
address of the contact author for your paper. The Contact Volume Editor
uses these details to compile a list for our production department at
SPS in India. Once the files have been worked upon, SPS sends a copy of
the final pdf of each paper to its contact author. The contact author is
asked to check through the final pdf to make sure that no errors have
crept in during the transfer or preparation of the files. This should
not be seen as an opportunity to update or copyedit the papers, which is
not possible due to time constraints. Only errors introduced during the
preparation of the files will be corrected.

This round of checking takes place about two weeks after the files have
been sent to the Editorial by the Contact Volume Editor, i.e., roughly
seven weeks before the start of the conference for conference
proceedings, or seven weeks before the volume leaves the printer's, for
post-proceedings. If SPS does not receive a reply from a particular
contact author, within the timeframe given, then it is presumed that the
author has found no errors in the paper. The tight publication schedule
of LNCS does not allow SPS to send reminders or search for alternative
email addresses on the Internet.

In some cases, it is the Contact Volume Editor that checks all the final
pdfs. In such cases, the authors are not involved in the checking phase.

\subsection{Additional Information Required by the Volume Editor}

If you have more than one surname, please make sure that the Volume Editor
knows how you are to be listed in the author index.

\subsection{Copyright Forms}

The copyright form may be downloaded from the ``For Authors"
(Information for LNCS Authors) section of the LNCS Website:
\texttt{www.springer.com/lncs}. Please send your signed copyright form
to the Contact Volume Editor, either as a scanned pdf or by fax or by
courier. One author may sign on behalf of all of the other authors of a
particular paper. Digital signatures are acceptable.

\section{Paper Preparation}

Springer provides you with a complete integrated \LaTeX{} document class
(\texttt{llncs.cls}) for multi-author books such as those in the LNCS
series. Papers not complying with the LNCS style will be reformatted.
This can lead to an increase in the overall number of pages. We would
therefore urge you not to squash your paper.

Please always cancel any superfluous definitions that are
not actually used in your text. If you do not, these may conflict with
the definitions of the macro package, causing changes in the structure
of the text and leading to numerous mistakes in the proofs.

If you wonder what \LaTeX{} is and where it can be obtained, see the
``\textit{LaTeX project site}'' (\url{http://www.latex-project.org})
and especially the webpage ``\textit{How to get it}''
(\url{http://www.latex-project.org/ftp.html}) respectively.

When you use \LaTeX\ together with our document class file,
\texttt{llncs.cls},
your text is typeset automatically in Computer Modern Roman (CM) fonts.
Please do
\emph{not} change the preset fonts. If you have to use fonts other
than the preset fonts, kindly submit these with your files.

Please use the commands \verb+\label+ and \verb+\ref+ for
cross-references and the commands \verb+\bibitem+ and \verb+\cite+ for
references to the bibliography, to enable us to create hyperlinks at
these places.

For preparing your figures electronically and integrating them into
your source file we recommend using the standard \LaTeX{} \verb+graphics+ or
\verb+graphicx+ package. These provide the \verb+\includegraphics+ command.
In general, please refrain from using the \verb+\special+ command.

Remember to submit any further style files and
fonts you have used together with your source files.

\subsubsection{Headings.}

Headings should be capitalized
(i.e., nouns, verbs, and all other words
except articles, prepositions, and conjunctions should be set with an
initial capital) and should,
with the exception of the title, be aligned to the left.
Words joined by a hyphen are subject to a special rule. If the first
word can stand alone, the second word should be capitalized.

Here are some examples of headings: ``Criteria to Disprove
Context-Freeness of Collage Language", ``On Correcting the Intrusion of
Tracing Non-deterministic Programs by Software", ``A User-Friendly and
Extendable Data Distribution System", ``Multi-flip Networks:
Parallelizing GenSAT", ``Self-determinations of Man".

\subsubsection{Lemmas, Propositions, and Theorems.}

The numbers accorded to lemmas, propositions, and theorems, etc. should
appear in consecutive order, starting with Lemma 1, and not, for
example, with Lemma 11.

\subsection{Figures}

For \LaTeX\ users, we recommend using the \emph{graphics} or \emph{graphicx}
package and the \verb+\includegraphics+ command.

Please check that the lines in line drawings are not
interrupted and are of a constant width. Grids and details within the
figures must be clearly legible and may not be written one on top of
the other. Line drawings should have a resolution of at least 800 dpi
(preferably 1200 dpi). The lettering in figures should have a height of
2~mm (10-point type). Figures should be numbered and should have a
caption which should always be positioned \emph{under} the figures, in
contrast to the caption belonging to a table, which should always appear
\emph{above} the table; this is simply achieved as matter of sequence in
your source.

Please center the figures or your tabular material by using the \verb+\centering+
declaration. Short captions are centered by default between the margins
and typeset in 9-point type (Fig.~\ref{fig:example} shows an example).
The distance between text and figure is preset to be about 8~mm, the
distance between figure and caption about 6~mm.

To ensure that the reproduction of your illustrations is of a reasonable
quality, we advise against the use of shading. The contrast should be as
pronounced as possible.

If screenshots are necessary, please make sure that you are happy with
the print quality before you send the files.
\begin{figure}
\centering
\includegraphics[height=6.2cm]{eijkel2}
\caption{One kernel at $x_s$ (\emph{dotted kernel}) or two kernels at
$x_i$ and $x_j$ (\textit{left and right}) lead to the same summed estimate
at $x_s$. This shows a figure consisting of different types of
lines. Elements of the figure described in the caption should be set in
italics, in parentheses, as shown in this sample caption.}
\label{fig:example}
\end{figure}

Please define figures (and tables) as floating objects. Please avoid
using optional location parameters like ``\verb+[h]+" for ``here".

\paragraph{Remark 1.}

In the printed volumes, illustrations are generally black and white
(halftones), and only in exceptional cases, and if the author is
prepared to cover the extra cost for color reproduction, are colored
pictures accepted. Colored pictures are welcome in the electronic
version free of charge. If you send colored figures that are to be
printed in black and white, please make sure that they really are
legible in black and white. Some colors as well as the contrast of
converted colors show up very poorly when printed in black and white.

\subsection{Formulas}

Displayed equations or formulas are centered and set on a separate
line (with an extra line or halfline space above and below). Displayed
expressions should be numbered for reference. The numbers should be
consecutive within each section or within the contribution,
with numbers enclosed in parentheses and set on the right margin --
which is the default if you use the \emph{equation} environment, e.g.,
\begin{equation}
  \psi (u) = \int_{o}^{T} \left[\frac{1}{2}
  \left(\Lambda_{o}^{-1} u,u\right) + N^{\ast} (-u)\right] dt \;  .
\end{equation}

Equations should be punctuated in the same way as ordinary
text but with a small space before the end punctuation mark.

\subsection{Footnotes}

The superscript numeral used to refer to a footnote appears in the text
either directly after the word to be discussed or -- in relation to a
phrase or a sentence -- following the punctuation sign (comma,
semicolon, or period). Footnotes should appear at the bottom of
the
normal text area, with a line of about 2~cm set
immediately above them.\footnote{The footnote numeral is set flush left
and the text follows with the usual word spacing.}

\subsection{Program Code}

Program listings or program commands in the text are normally set in
typewriter font, e.g., CMTT10 or Courier.

\medskip

\noindent
{\it Example of a Computer Program}
\begin{verbatim}
program Inflation (Output)
  {Assuming annual inflation rates of 7%, 8%, and 10%,...
   years};
   const
     MaxYears = 10;
   var
     Year: 0..MaxYears;
     Factor1, Factor2, Factor3: Real;
   begin
     Year := 0;
     Factor1 := 1.0; Factor2 := 1.0; Factor3 := 1.0;
     WriteLn('Year  7% 8% 10%'); WriteLn;
     repeat
       Year := Year + 1;
       Factor1 := Factor1 * 1.07;
       Factor2 := Factor2 * 1.08;
       Factor3 := Factor3 * 1.10;
       WriteLn(Year:5,Factor1:7:3,Factor2:7:3,Factor3:7:3)
     until Year = MaxYears
end.
\end{verbatim}
%
\noindent
{\small (Example from Jensen K., Wirth N. (1991) Pascal user manual and
report. Springer, New York)}

\subsection{Citations}

For citations in the text please use
square brackets and consecutive numbers: \cite{jour}, \cite{lncschap},
\cite{proceeding1} -- provided automatically
by \LaTeX 's \verb|\cite| \dots\verb|\bibitem| mechanism.

\subsection{Page Numbering and Running Heads}

There is no need to include page numbers. If your paper title is too
long to serve as a running head, it will be shortened. Your suggestion
as to how to shorten it would be most welcome.

\section{LNCS Online}

The online version of the volume will be available in LNCS Online.
Members of institutes subscribing to the Lecture Notes in Computer
Science series have access to all the pdfs of all the online
publications. Non-subscribers can only read as far as the abstracts. If
they try to go beyond this point, they are automatically asked, whether
they would like to order the pdf, and are given instructions as to how
to do so.

Please note that, if your email address is given in your paper,
it will also be included in the meta data of the online version.

\section{BibTeX Entries}

The correct BibTeX entries for the Lecture Notes in Computer Science
volumes can be found at the following Website shortly after the
publication of the book:
\url{http://www.informatik.uni-trier.de/~ley/db/journals/lncs.html}

\subsubsection*{Acknowledgments.} The heading should be treated as a
subsubsection heading and should not be assigned a number.

\section{The References Section}\label{references}

In order to permit cross referencing within LNCS-Online, and eventually
between different publishers and their online databases, LNCS will,
from now on, be standardizing the format of the references. This new
feature will increase the visibility of publications and facilitate
academic research considerably. Please base your references on the
examples below. References that don't adhere to this style will be
reformatted by Springer. You should therefore check your references
thoroughly when you receive the final pdf of your paper.
The reference section must be complete. You may not omit references.
Instructions as to where to find a fuller version of the references are
not permissible.

We only accept references written using the latin alphabet. If the title
of the book you are referring to is in Russian or Chinese, then please write
(in Russian) or (in Chinese) at the end of the transcript or translation
of the title.

The following section shows a sample reference list with entries for
journal articles \cite{jour}, an LNCS chapter \cite{lncschap}, a book
\cite{book}, proceedings without editors \cite{proceeding1} and
\cite{proceeding2}, as well as a URL \cite{url}.
Please note that proceedings published in LNCS are not cited with their
full titles, but with their acronyms!

\begin{thebibliography}{4}

\bibitem{wang} Wang, J., Hu, C., Zhang, J., Li, J.: OpenMP compiler for distributed memory architectures. Science China Information Sciences. Springer Publishing. 53 (5): 932–944. (2010)

\bibitem{basumallik} Basumallik, A., Min, S., Eigenmann, R.: Programming Distributed Memory Sytems [sic] using OpenMP. Proceedings of the 2007 IEEE International Parallel and Distributed Processing Symposium. New York: IEEE Press. (2007)

\bibitem{costa} Costa, J.J.; et al.: Running OpenMP applications efficiently on an everything-shared SDSM. Journal of Parallel and Distributed Computing. Elsevier. 66 (5), 647–658 (2006)

\bibitem{gagne} Gagne, A.S., Peter Baer Galvin, G.: Operating system concepts (9th ed.). Hoboken, N.J.: Wiley. pp. 181–182. 

\bibitem{kern} Kernighan, B.W., Ritchie, D.M.: The C Programming Language (1st ed.). Englewood Cliffs, NJ: Prentice Hall. (1978)

\bibitem{nesterov} Nesterov Ju. E. Metody vypukloj optimizacii. — M.: Izdatel'stvo MCNMO, 2010. — 281 s.

\bibitem{jour} Smith, T.F., Waterman, M.S.: Identification of Common Molecular
Subsequences. J. Mol. Biol. 147, 195--197 (1981)

\bibitem{lncschap} May, P., Ehrlich, H.C., Steinke, T.: ZIB Structure Prediction Pipeline:
Composing a Complex Biological Workflow through Web Services. In: Nagel,
W.E., Walter, W.V., Lehner, W. (eds.) Euro-Par 2006. LNCS, vol. 4128,
pp. 1148--1158. Springer, Heidelberg (2006)

\bibitem{book} Foster, I., Kesselman, C.: The Grid: Blueprint for a New Computing
Infrastructure. Morgan Kaufmann, San Francisco (1999)

\bibitem{proceeding1} Czajkowski, K., Fitzgerald, S., Foster, I., Kesselman, C.: Grid
Information Services for Distributed Resource Sharing. In: 10th IEEE
International Symposium on High Performance Distributed Computing, pp.
181--184. IEEE Press, New York (2001)

\bibitem{proceeding2} Foster, I., Kesselman, C., Nick, J., Tuecke, S.: The Physiology of the
Grid: an Open Grid Services Architecture for Distributed Systems
Integration. Technical report, Global Grid Forum (2002)

\bibitem{url} National Center for Biotechnology Information, \url{http://www.ncbi.nlm.nih.gov}

\end{thebibliography}


\section*{Appendix: Springer-Author Discount}

LNCS authors are entitled to a 33.3\% discount off all Springer
publications. Before placing an order, the author should send an email,
giving full details of his or her Springer publication,
to \url{orders-HD-individuals@springer.com} to obtain a so-called token. This token is a
number, which must be entered when placing an order via the Internet, in
order to obtain the discount.

\section{Checklist of Items to be Sent to Volume Editors}
Here is a checklist of everything the volume editor requires from you:


\begin{itemize}
\settowidth{\leftmargin}{{\Large$\square$}}\advance\leftmargin\labelsep
\itemsep8pt\relax
\renewcommand\labelitemi{{\lower1.5pt\hbox{\Large$\square$}}}

\item The final \LaTeX{} source files
\item A final PDF file
\item A copyright form, signed by one author on behalf of all of the
authors of the paper.
\item A readme giving the name and email address of the
corresponding author.
\end{itemize}
\end{document}
